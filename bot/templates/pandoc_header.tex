\usepackage{amsmath}
\usepackage{amssymb}
\usepackage{amsfonts}
\usepackage{graphicx}
\usepackage{mathrsfs}
\usepackage{xcolor}
\usepackage{newunicodechar}
\usepackage{mathtools}
\usepackage{longtable}
\usepackage{fontspec}
\usepackage[main=russian]{babel}
\usepackage{microtype}

% --- КЛЮЧЕВОЕ ИСПРАВЛЕНИЕ №1: Явно задаем шрифт для МАТЕМАТИКИ ---
% Это решает проблему "Missing character" для математических символов.
\usepackage{unicode-math}
\setmathfont{DejaVu Math TeX Gyre} % Этот шрифт есть в вашем Docker-образе

% --- Явная настройка шрифтов для ТЕКСТА (кириллицы) ---
\babelfont{rm}[
  Ligatures=TeX,
  BoldFont = DejaVu Serif Bold,
  ItalicFont = DejaVu Serif Italic,
  BoldItalicFont = DejaVu Serif Bold Italic
]{DejaVu Serif}
\babelfont{sf}[
  Ligatures=TeX,
  BoldFont = DejaVu Sans Bold,
  ItalicFont = DejaVu Sans Oblique,
  BoldItalicFont = DejaVu Sans Bold Oblique
]{DejaVu Sans}
\babelfont{tt}{DejaVu Sans Mono}

\newunicodechar{∂}{\partial}
\newunicodechar{Δ}{\Delta}

% --- КЛЮЧЕВОЕ ИСПРАВЛЕНИЕ №2: Более надежная настройка hyperref ---
% Загружаем одним из последних и исправляем \hypertarget для кириллицы
\usepackage{hyperref}
\hypersetup{
    unicode=true,
    pdftitle={\@title},
    pdfauthor={\@author},
    colorlinks=true,
    linkcolor=blue,
    citecolor=green,
    urlcolor=magenta,
    breaklinks=true
}
\makeatletter
% Pandoc использует \hypertarget для создания якорей. Без этого исправления
% кириллические заголовки ломают команду, вызывая ошибки `nullfont`.
\let\oldhypertarget\hypertarget
\renewcommand{\hypertarget}[2]{\oldhypertarget{#1}{\texorpdfstring{#2}{#2}}}
\makeatother